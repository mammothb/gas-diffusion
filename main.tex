\documentclass[8pt, a4paper]{article}
% packages and command files
\usepackage{amsmath}
\usepackage{array}
\usepackage{bm}
\usepackage[margin=0.25in]{geometry}


\newlength{\mycolwidth}
\settowidth{\mycolwidth}{$-2+(1+a_{n-1})$} % widest entry
% definitions
\newcommand{\CNO}{\ensuremath{C_{NO}}}
\newcommand{\DNO}{\ensuremath{D_{NO}}}
\newcommand{\RNO}{\ensuremath{R_{NO}}}
\newcommand{\dr}{\ensuremath{\Delta r}}
\newcommand{\ubf}{\ensuremath{\mathbf{\vec{u}}}}

\begin{document}
  Starting equations
  \begin{equation}
  \frac{\DNO}{r} \frac{d}{dr} \left (r \frac{d\CNO}{dr} \right) + \RNO = 0
  \end{equation}
  
  Simplify
  \begin{equation}
  \DNO \frac{d^2\CNO}{dr^2} + \frac{\DNO}{r} \frac{d\CNO}{dr} = - \RNO
  \end{equation}
  
  Compare to form
  \begin{equation} \label{eq:form}
  u''+P(r)u'=F(r)
  \end{equation}
  \begin{equation}
  \:u=\CNO,\:P(r)=\frac{1}{r},\:F(r)=-\frac{\RNO}{\DNO}
  \end{equation}
  
  Taylor expansion
  \begin{align*}
  u_{i+1}&=u_i+u_i'\dr+\frac{1}{2}u_i''\dr^2,\\
  u_{i-1}&=u_i-u_i'\dr+\frac{1}{2}u_i''\dr^2
  \end{align*}
  
  Central difference
  \begin{align*}
  u_i'&=\frac{u_{i+1}-u_{i-1}}{2\dr},\\
  u_i''&=\frac{u_{i+1}-2u_i+u_{i-1}}{\dr^2}
  \end{align*}
  
  Sub back into Eq. \eqref{eq:form}
  \begin{equation}
  \frac{u_{i+1}-2u_i+u_{i-1}}{h^2}+P_i\frac{u_{i+1}-u_{i-1}}{2h}=F_i,\:h=\dr
  \end{equation}
  
  where index $i$ starts at 1. 
  
  Rearrange
  \begin{equation}
  \left (1-\frac{h}{2}P_i \right)u_{i-1}+(-2)u_i+\left (1+\frac{h}{2}P_i 
  \right)u_{i+1}=h^2F_i
  \end{equation}

  Split into 4 sections since equations for vascular wall and tissue are the 
  same. Solutions are in the general form $A\ubf=B$. Let $a_i=\frac{h}{2}P_i$.

  RBC core $r_0 < r < r_1$
  \begin{align*}
  u'(0)&=0\rightarrow u_1 = u_2,\\
  u_{RBC}'(r_1)&=\sigma,\:\sigma=u_{CFL}'(r_1)
  \end{align*}
  Using second-order accurate one-sided difference approximation, $j$ 
  represents indexes in CFL domain.
  \begin{align*}
  \sigma&=\frac{-3u_j+4u_{j+1}-u_{j+2}}{2h},\\
  \frac{3u_i-4u_{i-1}+u_{i-2}}{2h}&=\sigma
  \end{align*}
  
  \setlength{\extrarowheight}{1.25\baselineskip}
  \begin{equation}
  A=
  \begin{bmatrix}
  -2-(1-a_2) & 1+a_2 & 0 & \cdots & \cdots & \cdots & \cdots & 0\\
  1-a_3 & -2 & 1+a_3 & \ddots & & & & \vdots\\
  0 & 1-a_4 & -2 & \ddots & \ddots & &  & \vdots\\
  \vdots & \ddots & \ddots & \ddots & \ddots & & & \vdots\\
  \vdots & & \ddots & \ddots & -2 & 1+a_{n-3} & 0 & 0 \\
  \vdots & & & \ddots & 1-a_{n-2} & -2 & 1+a_{n-2} & 0\\
  \vdots & \cdots & \cdots & \cdots & 0 & 1-a_{n-1} & -2 & 1+a_{n-1}\\
  0 & \cdots & \cdots & \cdots & \cdots & \frac{h}{2} & -2h & 
  \frac{3h}{2}\\[5ex]
  \end{bmatrix},\:
  \ubf=
  \begin{bmatrix}
  u_2\\
  u_3\\
  \vdots\\
  u_i\\
  \vdots\\
  u_n
  \end{bmatrix},\:
  B=
  \begin{bmatrix}
  h^2F_2\\
  h^2F_3\\
  \vdots\\
  h^2F_i\\
  \vdots\\
  h^2F_{n-1}\\
  \sigma
  \end{bmatrix}
  \end{equation}

  CFL $r_1<r<r_2$
  \begin{align*}
  u_{CFL}'(r_1)&=\sigma,\:\sigma=u_{RBC}'(r_1),\\
  u_{CFL}'(r_2)&=\phi,\:\phi=u_{EC}'(r_2),\\
  \end{align*}
  
  Using one-sided difference approximation again, $i$ is CFL, $j$ is RBC and 
  $k$ is EC.
  \begin{align*}
  \sigma&=\frac{3u_j-4u_{j-1}+u_{j-2}}{2h},\\
  \frac{-3u_i+4u_{i+1}-u_{i+2}}{2h}&=\sigma,\\
  \phi&=\frac{-3u_k+4u_{k+1}-u_{k+2}}{2h},\\
  \frac{3u_i-4u_{i-1}+u_{i-2}}{2h}&=\phi
  \end{align*}
  \begin{equation}
  A=
  \begin{bmatrix}
  -\frac{3h}{2} & 2h & -\frac{h}{2} & 0 & \cdots & \cdots & \cdots & \cdots & 
  0\\
  0 & -2 & 1+a_2 & 0 & & & & & \vdots\\
  \vdots & 1-a_3 & -2 & 1+a_3 & \ddots & & & & \vdots\\
  \vdots & 0 & 1-a_4 & -2 & \ddots & \ddots & &  & \vdots\\
  \vdots & & \ddots & \ddots & \ddots & \ddots & & & \vdots\\
  \vdots & & & \ddots & \ddots & -2 & 1+a_{n-3} & 0 & 0\\
  \vdots & & & & \ddots & 1-a_{n-2} & -2 & 1+a_{n-2} & 0\\
  \vdots & \cdots & \cdots & \cdots & \cdots & 0 & 1-a_{n-1} & -2 & 1+a_{n-1}\\
  0 & \cdots & \cdots & \cdots & \cdots & \cdots & \frac{h}{2} & -2h & 
  \frac{3h}{2}\\[5ex]
  \end{bmatrix},\:
  \ubf=
  \begin{bmatrix}
  u_1\\
  u_2\\
  u_3\\
  \vdots\\
  u_i\\
  \vdots\\
  u_n
  \end{bmatrix},\:
  B=
  \begin{bmatrix}
  \sigma\\
  h^2F_2\\
  h^2F_3\\
  \vdots\\
  h^2F_i\\
  \vdots\\
  h^2F_{n-1}\\
  \phi
  \end{bmatrix}
  \end{equation}
  
  EC $r_2<r<r_3$ is same method as CFL, only difference is in the \RNO term.
  \newpage
  VW and T $r_3<r<r_5$. Combined since same governing equation.
    \begin{align*}
    u_{VW}'(r_3)&=\gamma,\:\gamma=u_{EC}'(r_3),\\    
    u'(r_5)&=0\rightarrow u_n=u_{n-1}
    \end{align*}
  Using one-sided difference approximation again, $i$ is VW\&T, $j$ is EC.
  \begin{align*}
  \phi&=\frac{3u_j-4u_{j-1}+u_{j-2}}{2h},\\
  \frac{-3u_i+4u_{i+1}-u_{i+2}}{2h}&=\phi
  \end{align*}
  \begin{equation}
  A=
  \begin{bmatrix}
  -\frac{3h}{2} & 2h & -\frac{h}{2} & 0 & \cdots & \cdots & \cdots & 0\\
  0 & -2 & 1+a_2 & 0 & & & & \vdots\\
  \vdots & 1-a_3 & -2 & 1+a_3 & \ddots & & & \vdots\\
  \vdots & 0 & 1-a_4 & -2 & \ddots & \ddots & & \vdots\\
  \vdots & & \ddots & \ddots & \ddots & \ddots & & \vdots\\
  \vdots & & & \ddots & \ddots & -2 & 1+a_{n-3} & 0 \\
  \vdots & & & & \ddots & 1-a_{n-2} & -2 & 1+a_{n-2} \\
  0 & \cdots & \cdots & \cdots & \cdots & 0 & 1-a_{n-1} & -2+(1+a_{n-1}) \\[5ex]
  \end{bmatrix},\:
  \ubf=
  \begin{bmatrix}
  u_1\\
  u_2\\
  u_3\\
  \vdots\\
  u_i\\
  \vdots\\
  u_{n-1}
  \end{bmatrix},\:
  B=
  \begin{bmatrix}
  \gamma\\
  h^2F_2\\
  h^2F_3\\
  \vdots\\
  h^2F_i\\
  \vdots\\
  h^2F_{n-1}
  \end{bmatrix}
  \end{equation}
\end{document}